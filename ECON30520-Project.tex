% Options for packages loaded elsewhere
% Options for packages loaded elsewhere
\PassOptionsToPackage{unicode}{hyperref}
\PassOptionsToPackage{hyphens}{url}
\PassOptionsToPackage{dvipsnames,svgnames,x11names}{xcolor}
%
\documentclass[
  letterpaper,
  DIV=11,
  numbers=noendperiod]{scrartcl}
\usepackage{xcolor}
\usepackage{amsmath,amssymb}
\setcounter{secnumdepth}{-\maxdimen} % remove section numbering
\usepackage{iftex}
\ifPDFTeX
  \usepackage[T1]{fontenc}
  \usepackage[utf8]{inputenc}
  \usepackage{textcomp} % provide euro and other symbols
\else % if luatex or xetex
  \usepackage{unicode-math} % this also loads fontspec
  \defaultfontfeatures{Scale=MatchLowercase}
  \defaultfontfeatures[\rmfamily]{Ligatures=TeX,Scale=1}
\fi
\usepackage{lmodern}
\ifPDFTeX\else
  % xetex/luatex font selection
\fi
% Use upquote if available, for straight quotes in verbatim environments
\IfFileExists{upquote.sty}{\usepackage{upquote}}{}
\IfFileExists{microtype.sty}{% use microtype if available
  \usepackage[]{microtype}
  \UseMicrotypeSet[protrusion]{basicmath} % disable protrusion for tt fonts
}{}
\makeatletter
\@ifundefined{KOMAClassName}{% if non-KOMA class
  \IfFileExists{parskip.sty}{%
    \usepackage{parskip}
  }{% else
    \setlength{\parindent}{0pt}
    \setlength{\parskip}{6pt plus 2pt minus 1pt}}
}{% if KOMA class
  \KOMAoptions{parskip=half}}
\makeatother
% Make \paragraph and \subparagraph free-standing
\makeatletter
\ifx\paragraph\undefined\else
  \let\oldparagraph\paragraph
  \renewcommand{\paragraph}{
    \@ifstar
      \xxxParagraphStar
      \xxxParagraphNoStar
  }
  \newcommand{\xxxParagraphStar}[1]{\oldparagraph*{#1}\mbox{}}
  \newcommand{\xxxParagraphNoStar}[1]{\oldparagraph{#1}\mbox{}}
\fi
\ifx\subparagraph\undefined\else
  \let\oldsubparagraph\subparagraph
  \renewcommand{\subparagraph}{
    \@ifstar
      \xxxSubParagraphStar
      \xxxSubParagraphNoStar
  }
  \newcommand{\xxxSubParagraphStar}[1]{\oldsubparagraph*{#1}\mbox{}}
  \newcommand{\xxxSubParagraphNoStar}[1]{\oldsubparagraph{#1}\mbox{}}
\fi
\makeatother


\usepackage{longtable,booktabs,array}
\usepackage{calc} % for calculating minipage widths
% Correct order of tables after \paragraph or \subparagraph
\usepackage{etoolbox}
\makeatletter
\patchcmd\longtable{\par}{\if@noskipsec\mbox{}\fi\par}{}{}
\makeatother
% Allow footnotes in longtable head/foot
\IfFileExists{footnotehyper.sty}{\usepackage{footnotehyper}}{\usepackage{footnote}}
\makesavenoteenv{longtable}
\usepackage{graphicx}
\makeatletter
\newsavebox\pandoc@box
\newcommand*\pandocbounded[1]{% scales image to fit in text height/width
  \sbox\pandoc@box{#1}%
  \Gscale@div\@tempa{\textheight}{\dimexpr\ht\pandoc@box+\dp\pandoc@box\relax}%
  \Gscale@div\@tempb{\linewidth}{\wd\pandoc@box}%
  \ifdim\@tempb\p@<\@tempa\p@\let\@tempa\@tempb\fi% select the smaller of both
  \ifdim\@tempa\p@<\p@\scalebox{\@tempa}{\usebox\pandoc@box}%
  \else\usebox{\pandoc@box}%
  \fi%
}
% Set default figure placement to htbp
\def\fps@figure{htbp}
\makeatother


% definitions for citeproc citations
\NewDocumentCommand\citeproctext{}{}
\NewDocumentCommand\citeproc{mm}{%
  \begingroup\def\citeproctext{#2}\cite{#1}\endgroup}
\makeatletter
 % allow citations to break across lines
 \let\@cite@ofmt\@firstofone
 % avoid brackets around text for \cite:
 \def\@biblabel#1{}
 \def\@cite#1#2{{#1\if@tempswa , #2\fi}}
\makeatother
\newlength{\cslhangindent}
\setlength{\cslhangindent}{1.5em}
\newlength{\csllabelwidth}
\setlength{\csllabelwidth}{3em}
\newenvironment{CSLReferences}[2] % #1 hanging-indent, #2 entry-spacing
 {\begin{list}{}{%
  \setlength{\itemindent}{0pt}
  \setlength{\leftmargin}{0pt}
  \setlength{\parsep}{0pt}
  % turn on hanging indent if param 1 is 1
  \ifodd #1
   \setlength{\leftmargin}{\cslhangindent}
   \setlength{\itemindent}{-1\cslhangindent}
  \fi
  % set entry spacing
  \setlength{\itemsep}{#2\baselineskip}}}
 {\end{list}}
\usepackage{calc}
\newcommand{\CSLBlock}[1]{\hfill\break\parbox[t]{\linewidth}{\strut\ignorespaces#1\strut}}
\newcommand{\CSLLeftMargin}[1]{\parbox[t]{\csllabelwidth}{\strut#1\strut}}
\newcommand{\CSLRightInline}[1]{\parbox[t]{\linewidth - \csllabelwidth}{\strut#1\strut}}
\newcommand{\CSLIndent}[1]{\hspace{\cslhangindent}#1}



\setlength{\emergencystretch}{3em} % prevent overfull lines

\providecommand{\tightlist}{%
  \setlength{\itemsep}{0pt}\setlength{\parskip}{0pt}}



 


\usepackage{float}
\usepackage{tabularray}
\usepackage[normalem]{ulem}
\usepackage{graphicx}
\UseTblrLibrary{booktabs}
\UseTblrLibrary{rotating}
\UseTblrLibrary{siunitx}
\NewTableCommand{\tinytableDefineColor}[3]{\definecolor{#1}{#2}{#3}}
\newcommand{\tinytableTabularrayUnderline}[1]{\underline{#1}}
\newcommand{\tinytableTabularrayStrikeout}[1]{\sout{#1}}
\KOMAoption{captions}{tableheading}
\makeatletter
\@ifpackageloaded{caption}{}{\usepackage{caption}}
\AtBeginDocument{%
\ifdefined\contentsname
  \renewcommand*\contentsname{Table of contents}
\else
  \newcommand\contentsname{Table of contents}
\fi
\ifdefined\listfigurename
  \renewcommand*\listfigurename{List of Figures}
\else
  \newcommand\listfigurename{List of Figures}
\fi
\ifdefined\listtablename
  \renewcommand*\listtablename{List of Tables}
\else
  \newcommand\listtablename{List of Tables}
\fi
\ifdefined\figurename
  \renewcommand*\figurename{Figure}
\else
  \newcommand\figurename{Figure}
\fi
\ifdefined\tablename
  \renewcommand*\tablename{Table}
\else
  \newcommand\tablename{Table}
\fi
}
\@ifpackageloaded{float}{}{\usepackage{float}}
\floatstyle{ruled}
\@ifundefined{c@chapter}{\newfloat{codelisting}{h}{lop}}{\newfloat{codelisting}{h}{lop}[chapter]}
\floatname{codelisting}{Listing}
\newcommand*\listoflistings{\listof{codelisting}{List of Listings}}
\makeatother
\makeatletter
\makeatother
\makeatletter
\@ifpackageloaded{caption}{}{\usepackage{caption}}
\@ifpackageloaded{subcaption}{}{\usepackage{subcaption}}
\makeatother
\usepackage{bookmark}
\IfFileExists{xurl.sty}{\usepackage{xurl}}{} % add URL line breaks if available
\urlstyle{same}
\hypersetup{
  pdftitle={The Determinants of Economics Students' Final Score},
  pdfauthor={Ronan Moran},
  colorlinks=true,
  linkcolor={blue},
  filecolor={Maroon},
  citecolor={Blue},
  urlcolor={Blue},
  pdfcreator={LaTeX via pandoc}}


\title{The Determinants of Economics Students' Final Score}
\author{Ronan Moran}
\date{}
\begin{document}
\maketitle


\subsection{1. Introduction}\label{introduction}

The intention of this journal entry is to find the determinants of
Students' final exam scores within the field of economics. It has been
found that the biggest contributors to Students final exam scores are
College GPA, high school GPA and ACT composite scores while factors such
as gender, attendance and parental education played an insignificant
role in contributing to student's final exam scores.

This paper is split into 3 main parts; Variable insight, which includes
a summary of variables and R functions used for the reader to gain an
intuition of the raw data. Next is our regression section, where OLS
methods and GAMs were used to determine the associated effects of each
variable on the dependent variable. Finally is a section dedicated to
machine learning, where the function random forests was used to
determine the importance of the various variables used in this study.

From conducting scholarly research, it was apparent that Gender, high
school GPA and education level of the parents were seen as the main
contributing factors to exam score (Erdem, Senturk, and Arslan (2008)) .
This paper seeks to investigate the validity of this claim through
various regression methods.

\subsection{2. Variable insight}\label{variable-insight}

\subsubsection{2.1 Variables and R
functions}\label{variables-and-r-functions}

The data was attained from an anonymous college in the United States.
There are 856 observations total, with 17 variables. The variables
collected are described in the
\phantomsection\label{list}\hyperref[list]{list of variables}. The
column `Variable code' describes the code which was assigned to each
variable, which will correspond with each beta value in our
Table~\ref{tbl-regressions}. The variable description clarifies the
detail of the variable, and the variable type briefly describes the role
of the variable in the study. Our variable of interest, i.e.~the
dependent variable of the study is `score'. Below,
Table~\ref{tbl-custom-2} lists the mean, standard deviation (SD), the
minimum value, the median value, and the max value of each of the
variables in the data set. It also has a histogram assigned to each
variable which shows the distribution of the sample relative to each
variable. In Table~\ref{tbl-custom-2}, the variables econhs, male,
calculus, attexc, attgood, fathcoll, mothcoll were all left out as
because they are binary variables, the information displayed would be
less insightful. Table~\ref{tbl-correlation} is a list of correlation
coefficients with the intention of challenging intuition before
regressing with the data. Figure~\ref{fig-patchwork} is a collection a
graphs plotting each binary variable against the final score of
economics students. Finally, Figure~\ref{fig-mutate} graphs `attbad', a
newly created variable (which is explained below), against the final
score.

\begin{longtable}[]{@{}
  >{\raggedright\arraybackslash}p{(\linewidth - 4\tabcolsep) * \real{0.3333}}
  >{\raggedright\arraybackslash}p{(\linewidth - 4\tabcolsep) * \real{0.3333}}
  >{\raggedright\arraybackslash}p{(\linewidth - 4\tabcolsep) * \real{0.3333}}@{}}
\caption{List of variables \{\#list\}.}\tabularnewline
\toprule\noalign{}
\begin{minipage}[b]{\linewidth}\raggedright
Variable code
\end{minipage} & \begin{minipage}[b]{\linewidth}\raggedright
Variable description
\end{minipage} & \begin{minipage}[b]{\linewidth}\raggedright
Variable type
\end{minipage} \\
\midrule\noalign{}
\endfirsthead
\toprule\noalign{}
\begin{minipage}[b]{\linewidth}\raggedright
Variable code
\end{minipage} & \begin{minipage}[b]{\linewidth}\raggedright
Variable description
\end{minipage} & \begin{minipage}[b]{\linewidth}\raggedright
Variable type
\end{minipage} \\
\midrule\noalign{}
\endhead
\bottomrule\noalign{}
\endlastfoot
age & Age in years & Independent, Continuous \\
work & Hours worked per week & Independent, Continuous \\
study & Hours studying per week & Independent, Continuous \\
econhs & If the student did economics in high school & Independent,
Factor; `Yes' indicates the student did economics in high school \\
colgpa & Student's college GPA at the beginning of the semester &
Independent, Ordinal; 0-4 \\
hsgpa & Student's high school GPA & Independent, Ordinal; 0-4 \\
acteng & Student's ACT English score & Independent, Ordinal; 1-36 \\
actmth & ACT math score & Independent, Ordinal; 1-36 \\
act & Student's composite ACT score & Independent, Ordinal; 1-36 \\
mathscr & Student's score on a math quiz & Independent,~Ordinal; 0-10 \\
male & Gender of the student & Independent, Factor; `male' indicates the
participant is male, `Female' indicates participant is female \\
calculus & If the student had taken a calculus course & Independent,
Factor; `Yes' indicates the participant took a calculus course \\
attexc & If the student's attendance was excellent & Independent,
Factor; `Yes' indicates the participants attendance was excellent \\
attgood & If the student's past attendance was good & Independent,
Factor; `Yes' indicates the participants attendance was good \\
fathcoll & If the student's father had a bachelors degree &
Independent,~Factor; `Yes' indicates that the participants father had a
bachelors degree \\
mothcoll & If student's mother had a bachelors degree & Independent,
Factor; `Yes' indicates that the participants mother had a bachelors
degree \\
score & Course score & Dependent, Ordinal; 0-100, in percent. \\
attbad & Student's attendance was less than `good' & Independent,
Factor; Yes indicates attendance was neither `good' nor `excellent' \\
\end{longtable}

The R function `dplyr' was used in our study to tidy our data. The
`mutate' function was used to convert binary variables into factor
variables. The purpose of this is to improve interpretation by
preventing R from treating it as a continuous numeric variable, instead
treating it as a categorical variable where one level (e.g.~`Yes')
becomes the reference level, and the coefficient represents the effect
of the other level, relative to the reference. Dplyr was also used to
arrange the data so that `score', our variable of interest, appears
first. Again, it was used to filter our the binary variables from
appearing on our summary table. A new variable called `attbad', using
the mutate function, was created, which indicates the student's
attendance was neither good nor excellent. This was done by assigning a
`yes' value if both `attexc' and `attgood' had a `No' value. The purpose
of this variable was to reveal how students with worse attendance than
`good' did on their final scores. Dplyr was also used to remove any rows
with `NA' values in order to keep our data consistent and usable with R
functions.

\subsubsection{2.2 Tables and graphs}\label{tables-and-graphs}

\begin{table}

\caption{\label{tbl-custom-2}Data summary table}

\centering{

\centering
\begin{tblr}[         %% tabularray outer open
]                     %% tabularray outer close
{                     %% tabularray inner open
colspec={Q[]Q[]Q[]Q[]Q[]Q[]Q[]Q[]Q[]},
column{1}={halign=l,},
column{2}={halign=l,},
column{3}={halign=l,},
column{4}={halign=l,},
column{5}={halign=l,},
column{6}={halign=l,},
column{7}={halign=l,},
column{8}={halign=l,},
column{9}={halign=l,},
}                     %% tabularray inner close
\toprule
& Unique & Missing Pct. & Mean & SD & Min & Median & Max & Histogram \\ \midrule %% TinyTableHeader
age & 9 & 0 & 19.4 & 0.9 & 18.0 & 19.0 & 29.0 & \includegraphics[height=1em]{tinytable_assets/idzpjga57gl7ai86cvcze5.png} \\
work & 44 & 0 & 8.6 & 9.2 & 0.0 & 8.0 & 37.5 & \includegraphics[height=1em]{tinytable_assets/idi47y1wiutekyshq0uvn0.png} \\
study & 52 & 0 & 13.9 & 7.8 & 0.0 & 12.0 & 50.0 & \includegraphics[height=1em]{tinytable_assets/idtiw5st1jm7akhez7sb7s.png} \\
colgpa & 612 & 0 & 2.8 & 0.5 & 0.9 & 2.8 & 4.0 & \includegraphics[height=1em]{tinytable_assets/iddiwv9o1ywpttyf4p9fpi.png} \\
hsgpa & 531 & 0 & 3.3 & 0.3 & 2.4 & 3.3 & 4.3 & \includegraphics[height=1em]{tinytable_assets/id7q43jm4wxba6576oplly.png} \\
acteng & 24 & 5 & 22.6 & 3.8 & 12.0 & 23.0 & 34.0 & \includegraphics[height=1em]{tinytable_assets/id98b2hlvhti0pwd0fs5fc.png} \\
actmth & 24 & 5 & 23.2 & 3.8 & 12.0 & 23.0 & 36.0 & \includegraphics[height=1em]{tinytable_assets/idcmkfpxvrmnr7iramer3r.png} \\
act & 21 & 5 & 23.1 & 3.3 & 13.0 & 23.0 & 33.0 & \includegraphics[height=1em]{tinytable_assets/ido1yhkosrcwco2xa33rog.png} \\
mathscr & 11 & 0 & 7.9 & 1.7 & 0.0 & 8.0 & 10.0 & \includegraphics[height=1em]{tinytable_assets/idd4gerbd0mwb2keu1zyu5.png} \\
score & 149 & 0 & 72.6 & 13.4 & 19.5 & 74.2 & 98.4 & \includegraphics[height=1em]{tinytable_assets/id25w0cxki005j13hkmvrz.png} \\
\bottomrule
\end{tblr}

}

\end{table}%

The summary table provides an overview of the key statistics for the
variables in the dataset. The ages of participants range from 18 to 29
years, with a mean age of 19.4 and a relatively small standard deviation
(SD) of 0.9, indicating most participants are close to the average age.
Hours worked per week vary significantly, ranging from 0 to 37.5 hours,
with a mean of 8.6 hours and a high SD of 9.2, suggesting substantial
variation in work hours. Study hours per week range from 0 to 50, with a
mean of 13.9 and an SD of 7.8. The wide range and higher SD indicate
diverse study habits among participants. College GPAs range from 0.9 to
4.0, with a mean of 2.8 and a low SD of 0.5, showing that most GPAs
cluster near the mean. High school GPAs range from 2.4 to 4.3, with a
mean of 3.3 and an SD of 0.3, indicating less variability than college
GPA. ACT English scores range from 12 to 34, with a mean of 22.6 and an
SD of 3.8, reflecting moderate variability in English performance. ACT
Math scores range from 12 to 36, with a mean of 23.2 and an SD of 3.8,
suggesting a similar variability as ACT English scores. Composite ACT
scores range from 13 to 33, with a mean of 23.1 and an SD of 3.3,
indicating consistent performance across participants. Math quiz scores
range from 0 to 10, with a mean of 7.9 and an SD of 1.7, showing
relatively high performance on the math quiz. Final course scores range
widely, from 19.5 to 98.4, with a mean of 72.6 and an SD of 13.4,
indicating significant variability in overall course performance. The
histograms visually summarize the distribution of each variable, showing
that variables like age, colgpa, and hsgpa are tightly clustered, while
others like work and study exhibit a wider spread.

\begin{table}

\caption{\label{tbl-correlation}Correlation between variables}

\centering{

\pandocbounded{\includegraphics[keepaspectratio]{ECON30520-Project_files/figure-pdf/tbl-correlation-1.pdf}}

}

\end{table}%

Table~\ref{tbl-correlation} displays the correlation coefficients and
also shows scatter plots (lower triangle) and density plots (diagonal)
for each pair of variables. The scatter plots illustrate how the
variables relate to one another visually: for instance, we can see an
upward ``cloud'' of points between final exam score and college GPA,
consistent with their relatively strong positive correlation of 0.57. A
similar upward trend is evident between final exam score and ACT math (r
= 0.41) and between final exam score and ACT composite (r = 0.39).
Meanwhile, variables like age display a weaker (and slightly negative)
relationship with the final exam score, reflected by a fairly diffuse
cloud of points and a small correlation (--0.07).\\
\strut \\
The density plots on the diagonal show how each variable is distributed
overall. For example, final exam scores appear roughly unimodal
(slightly skewed), while college GPA, ACT math, and ACT composite each
cluster around a particular central value with varying spreads. In
short, these visual patterns reinforce our numeric correlations: higher
GPAs and stronger ACT scores tend to line up with higher final exam
scores in economics. The regression analysis below will test whether
these relationships hold once other factors are accounted for.

\begin{figure}

\centering{

\pandocbounded{\includegraphics[keepaspectratio]{ECON30520-Project_files/figure-pdf/fig-patchwork-1.pdf}}

}

\caption{\label{fig-patchwork}Graph of factor variables}

\end{figure}%

The above graphs were created seeking to gain clarity on the
relationship between each of the variables and final economics exam
scores. Across all plots, there appears to be no substantial difference
in the distribution of scores between the `no' and `yes' categories. All
show similar spreads and central tendencies. This suggests minimal or no
strong relationship with scores. These graphs imply that the variables
investigated may not be strong predictors of student performance.
Further statistical testing via regression analysis is required to
verify these claims.

\begin{figure}

\centering{

\pandocbounded{\includegraphics[keepaspectratio]{ECON30520-Project_files/figure-pdf/fig-mutate-1.pdf}}

}

\caption{\label{fig-mutate}Graph of `attbad'}

\end{figure}%

Figure~\ref{fig-mutate} visualises the relationship between the factor
variable that was created, `attbad' and the dependent variable `score'.
`No' indicates students whose attendance was at least `good' while `yes'
indicates the student's attendance was at least worse than `good'. The
Y-axis represents our `score' variable. The median, which is the
horizontal line in the box, is slightly higher for students with `bad'
attendance. This suggests that, on average, students with bad attendance
have a slightly higher score. It should be noted however, that the
difference does not appear to be dramatic. The interquartile range,
i.e.~the height of the box, is slightly narrower for students with bad
attendance. This suggests there is more variability in students with
good or excellent attendance, however, this could be due to the sample
size being smaller in those with `bad' attendance. For students with
`good' or better attendance, there are several outliers below the 40\%
mark, meaning some students who had good or better attendance still
performed poorly on their final economics exams. This plot challenges
the intuitive assumption that bad attendance automatically lead to worse
scores on exams. As we can see, the difference is slight. `attbad' will
be included in a regression to investigate the implications of this
plot.

\subsection{3. Regressions and analysis}\label{regressions-and-analysis}

\subsubsection{3.1 Regression Methods}\label{regression-methods}

In this section our chosen regression method is ordinary least squares
(OLS). This method seeks to find the effect of the independent variable,
denoted as `beta', by minimizing the sum of squared residuals.

In matrix form, the OLS estimate is:

\[
\hat{\boldsymbol{\beta}} = (\mathbf{X}^\top \mathbf{X})^{-1} \mathbf{X}^\top \mathbf{y}
\]

where:

\begin{itemize}
\item
  \(\mathbf{X}\) is the design matrix of predictors,
\item
  \(\mathbf{y}\) is the vector of outcomes,
\item
  \(\hat{\boldsymbol{\beta}}\) is the vector of estimated coefficients.
\end{itemize}

An important assumption of OLS to note:

\[
x_j, \mu = 0 
\] where:

\begin{itemize}
\item
  \(x_j\) represents the \(j\)-th predictor,
\item
  \(\mu\) is the error term,
\item
  \(\mathbb{E}[\mu \mid X] = 0\) means the error term has a mean of zero
  given the predictors.
\end{itemize}

This means our Beta term and dependent variable must not both be
correlated with our error term, or our correlation coefficients will be
`biased' meaning, on average, our coefficients will be incorrect. By
satisfying this assumption, we ensure that our betas show the accurate
effect on the dependent variable on average.

The predicted value of \(Y\), denoted as \(\hat{y}\), is calculated as:

\(\hat{y} = \hat{\alpha} + \hat{\beta}_0 + \hat{\beta}_1 x_1 + \hat{\beta}_2 x_2 + \dots + \hat{\beta}_k x_k +\mu\)

\begin{itemize}
\item
  \(\hat{\mathbf{y}}\): Vector of predicted values,
\item
  \(\mathbf{X}\): Design matrix of predictors,
\item
  \(\hat{\boldsymbol{\beta}}\): Vector of estimated regression
  coefficients.
\item
  \(\boldsymbol{\mu}\): Vector of error terms.
\end{itemize}

Using this method, we will be able to speculate the associated effect of
beta on our dependent variable on average.

The method of selecting variables for the regression used was adding
each variable one by one. If the \(R^2\) value decreased, it meant the
model was a worse fit of the data, and it was removed. The variables
left out from this method were; age, attexc, attgood, fathcoll,
mothcoll, attbad. `attmath' and `atteng' were also left out in order to
avoid multicollinearity, which would cause our model to lose predictive
value. These variables were removed because they were highly correlated
with `act' which is a function of both of these, and other scores.

\subsubsection{3.2 Regressions and
Interpretations}\label{regressions-and-interpretations}

\begin{table}

\caption{\label{tbl-regressions}Regressions}

\centering{

\centering
\begin{talltblr}[         %% tabularray outer open
entry=none,label=none,
note{}={+ p \num{< 0.1}, * p \num{< 0.05}, ** p \num{< 0.01}, *** p \num{< 0.001}},
note{ }={Standard Errors are in parentheses. Logs not applicable to our data. Gender coefficient is true when gender = male. Calculus variable is true if student has taken calculus.},
]                     %% tabularray outer close
{                     %% tabularray inner open
colspec={Q[]Q[]Q[]Q[]Q[]Q[]Q[]Q[]Q[]},
column{1}={halign=l,},
column{2}={halign=c,},
column{3}={halign=c,},
column{4}={halign=c,},
column{5}={halign=c,},
column{6}={halign=c,},
column{7}={halign=c,},
column{8}={halign=c,},
column{9}={halign=c,},
hline{18}={1,2,3,4,5,6,7,8,9}{solid, 0.05em, black},
}                     %% tabularray inner close
\toprule
& College GPA & Add HS GPA & Add Gender & Add Hours studied & Add hours worked & Add ACT Score & Add Math Score & Add If Calculus Taken \\ \midrule %% TinyTableHeader
College GPA           & \num{14.323}*** & \num{12.667}*** & \num{12.713}*** & \num{12.734}*** & \num{12.398}*** & \num{11.556}*** & \num{11.489}*** & \num{11.327}*** \\
& (\num{0.705})   & (\num{0.799})   & (\num{0.785})   & (\num{0.786})   & (\num{0.791})   & (\num{0.786})   & (\num{0.784})   & (\num{0.775})   \\
HS GPA                &                  & \num{5.344}***  & \num{6.138}***  & \num{6.173}***  & \num{6.416}***  & \num{3.754}**   & \num{3.365}**   & \num{3.108}*    \\
&                  & (\num{1.254})   & (\num{1.242})   & (\num{1.243})   & (\num{1.240})   & (\num{1.289})   & (\num{1.294})   & (\num{1.279})   \\
Gender(male)          &                  &                  & \num{4.042}***  & \num{3.943}***  & \num{3.620}***  & \num{3.127}***  & \num{2.927}***  & \num{2.495}***  \\
&                  &                  & (\num{0.747})   & (\num{0.755})   & (\num{0.760})   & (\num{0.748})   & (\num{0.750})   & (\num{0.746})   \\
Hours Studied P/W     &                  &                  &                  & \num{-0.042}    & \num{-0.046}    & \num{0.006}     & \num{0.004}     & \num{0.003}     \\
&                  &                  &                  & (\num{0.048})   & (\num{0.048})   & (\num{0.048})   & (\num{0.047})   & (\num{0.047})   \\
Hours Worked P/W      &                  &                  &                  &                  & \num{-0.118}**  & \num{-0.132}*** & \num{-0.131}**  & \num{-0.131}*** \\
&                  &                  &                  &                  & (\num{0.041})   & (\num{0.040})   & (\num{0.040})   & (\num{0.039})   \\
Act Composite Score   &                  &                  &                  &                  &                  & \num{0.753}***  & \num{0.675}***  & \num{0.653}***  \\
&                  &                  &                  &                  &                  & (\num{0.124})   & (\num{0.127})   & (\num{0.126})   \\
Math score            &                  &                  &                  &                  &                  &                  & \num{0.565}*    & \num{0.268}     \\
&                  &                  &                  &                  &                  &                  & (\num{0.225})   & (\num{0.231})   \\
Taken calculus? (Yes) &                  &                  &                  &                  &                  &                  &                  & \num{3.763}***  \\
&                  &                  &                  &                  &                  &                  &                  & (\num{0.813})   \\
Num.Obs.              & \num{814}       & \num{814}       & \num{814}       & \num{814}       & \num{814}       & \num{814}       & \num{814}       & \num{814}       \\
R2                    & \num{0.337}     & \num{0.351}     & \num{0.374}     & \num{0.375}     & \num{0.381}     & \num{0.408}     & \num{0.413}     & \num{0.428}     \\
R2 Adj.               & \num{0.336}     & \num{0.350}     & \num{0.372}     & \num{0.372}     & \num{0.377}     & \num{0.404}     & \num{0.408}     & \num{0.423}     \\
AIC                   & \num{6194.0}    & \num{6178.0}    & \num{6151.0}    & \num{6152.3}    & \num{6145.8}    & \num{6111.2}    & \num{6106.8}    & \num{6087.4}    \\
\bottomrule
\end{talltblr}

}

\end{table}%

Table~\ref{tbl-regressions} show the results of several OLS regression.
College GPA has a consistently strong positive association with final
scores across all of the models. A one unit increase in college GPA is
associated with an increase of 11.3 points, average, in exam scores when
controlling for other variables. This coefficient is significant at the
99\% level. This indicates that college academic performance is a
critical determinant of success in economics exams.

High school GPA also displays a positive relationship with final exam
score. A one unit increase in High School GPA is associated with an
increase of 5.3 points, on average, in exam scores when only College GPA
is included in the regression. Interestingly, adding gender, hours
studied and hours worked to this regression increases this coefficient
to 6.4 points on average, indicating that these variables are negatively
correlated with each other. Then, adding ACT score, math score and if
the student took calculus brings the coefficient down to 3.1 points on
average, indicating that these variables are positively correlated with
each other. Although, this is only significant at the 90\% level.

The variable `Gender' indicates that being male is associated with an
increase in final score of 2.5 points, on average, compared to females
when all other variables are controlled for. This is statistically
significant at the 99\% confidence level.

Hours studied per week show no statistically significant relationship
with final scores, as the coefficient is near zero across all models.

Hours worked per week have a significant negative relationship with exam
scores. A one-hour increase in work is associated with a decrease in
exam scores by .13 points on average when controlling for all other
variables. This suggests that work commitments hinder academic
performance.

Math score shows a weak coefficient; a one unit increase in math score
is associated with a an increase in final exam score of .57 on average
when controlling for all variables except the calculus variable. In this
regression, it is significant at the 90\% level. However, when
controlling for if the student has taken calculus, this associated
effect becomes weaker and loses statistical significance.

Students who took calculus are associated with an increased score of
3.76 points on average than those who did not, with a significance level
at 99\%.

The R squared and adjusted R squares values increase as more variables
are added, suggesting that the model improved in fit and that no
unnecessary variables were added. The final value is .423, meaning the
model explains 42.3\% of the variance in final exam scores.

\begin{figure}

\centering{

\pandocbounded{\includegraphics[keepaspectratio]{ECON30520-Project_files/figure-pdf/fig-modelplot1-1.pdf}}

}

\caption{\label{fig-modelplot1}Model plot of Regressions so far}

\end{figure}%

Figure~\ref{fig-modelplot1} summarises the coefficients relative to one
another. It also displays their 95\% confidence intervals, allowing us
to compare how the inclusion of different variables affects the
estimated effects of predictors on economic student's final exam scores.

Our main takeaway from these regressions are; Academic preparations such
as college GPA, HS GPA, ACT composite score and calculus high indicators
of exam performance. External factors like work commitments are
associated with negative impacts on performance. Non-significant
variables like hours studied per week warrant further investigation to
understand their role in this data set.

\subsubsection{3.3 Regression Diagnostics}\label{regression-diagnostics}

Figure~\ref{fig-diagplot} below assess the assumptions of a linear
regression model. Each plot assesses the model's performance and whether
key assumptions are violated.

\begin{figure}

\centering{

\pandocbounded{\includegraphics[keepaspectratio]{ECON30520-Project_files/figure-pdf/fig-diagplot-1.pdf}}

}

\caption{\label{fig-diagplot}Diagnostic plots}

\end{figure}%

The purpose of the residuals vs fitted plot (top left) is to check for
non-linearity and constant variance, or homoskedasticity. The residuals
are scattered around zero, however there is a slight curved pattern in
the residuals indicating possible non-linearity in the relationship
between predictors and the outcome. The spread of residuals is quite
even, suggesting no strong evidence of heteroskedasticity (non constant
variance).

The purpose of the normal q-q plot (top right) is to check whether the
residuals follow a normal distribution. Most points lie close to the
diagonal line, suggesting that residuals are normal. There are some
deviations at the tails, particularly observation 293, which may
indicate potential outliers.

The purpose of the scale-location plot is to check for homoskedasticity.
The trend line is relatively and the spread of the points appears fairly
consistent across fitted values. However, the downward slope of the
trend line suggests a slight decrease in variance as fitted values
increase, indicating mild heteroskedasticity.

The residuals vs Leverage plot (bottom left) identifies influential
points. Most point have low leverage, as indicated by their clustering
near the left side of the plot. Observation 293 has higher leverage and
could potentially be an influential point. Its standardised residual
does not appear extreme, however, so its influence may be limited.

From our findings we can conclude that the linearity assumption may be
slightly violated due to the curve in the residuals vs fitted plot.
There also may be mild heterskedasticity, but no severe violations. This
means our coefficients may be slightly wrong.

Observation 293 is removed from the model in Figure~\ref{fig-diagplot2}
to analyse the difference to determine its effect.

A general additive model is applies in Figure~\ref{fig-regressionsGAM}
to address the possible non-linearity of our OLS regression.

\begin{table}

\caption{\label{tbl-regressions2}Regressions with and without leverage
point}

\centering{

\centering
\begin{talltblr}[         %% tabularray outer open
entry=none,label=none,
note{}={+ p \num{< 0.1}, * p \num{< 0.05}, ** p \num{< 0.01}, *** p \num{< 0.001}},
note{ }={Standard Errors are in parentheses. Logs not applicable to our data. Gender coefficient is true when gender = male. Calculus variable is true if student has taken calculus.},
]                     %% tabularray outer close
{                     %% tabularray inner open
colspec={Q[]Q[]Q[]},
column{1}={halign=l,},
column{2}={halign=c,},
column{3}={halign=c,},
hline{18}={1,2,3}{solid, 0.05em, black},
}                     %% tabularray inner close
\toprule
& Highest leveredge point left in & Leveredge point taken out \\ \midrule %% TinyTableHeader
College GPA           & \num{11.327}*** & \num{11.505}*** \\
& (\num{0.775})   & (\num{0.773})   \\
HS GPA                & \num{3.108}*    & \num{2.849}*    \\
& (\num{1.279})   & (\num{1.276})   \\
Gender(male)          & \num{2.495}***  & \num{2.550}***  \\
& (\num{0.746})   & (\num{0.743})   \\
Hours Studied P/W     & \num{0.003}     & \num{0.002}     \\
& (\num{0.047})   & (\num{0.047})   \\
Hours Worked P/W      & \num{-0.131}*** & \num{-0.133}*** \\
& (\num{0.039})   & (\num{0.039})   \\
Act Composite Score   & \num{0.653}***  & \num{0.639}***  \\
& (\num{0.126})   & (\num{0.125})   \\
Math score            & \num{0.268}     & \num{0.340}     \\
& (\num{0.231})   & (\num{0.231})   \\
Taken calculus? (Yes) & \num{3.763}***  & \num{3.630}***  \\
& (\num{0.813})   & (\num{0.810})   \\
Num.Obs.              & \num{814}       & \num{813}       \\
R2                    & \num{0.428}     & \num{0.433}     \\
R2 Adj.               & \num{0.423}     & \num{0.427}     \\
\bottomrule
\end{talltblr}

}

\end{table}%

\begin{figure}

\centering{

\pandocbounded{\includegraphics[keepaspectratio]{ECON30520-Project_files/figure-pdf/fig-diagplot2-1.pdf}}

}

\caption{\label{fig-diagplot2}Diagnostic plots with and without leverage
point}

\end{figure}%

From Table~\ref{tbl-regressions2} we can see that the variables has
slight change when taking out observation 293. However, as we can see
from Figure~\ref{fig-diagplot2} the difference is mild and does not
effect the overall takeaway of the regression.

\begin{figure}

\centering{

\pandocbounded{\includegraphics[keepaspectratio]{ECON30520-Project_files/figure-pdf/fig-regressionsGAM-1.pdf}}

}

\caption{\label{fig-regressionsGAM}GAM regression smooth terms}

\end{figure}%

Figure~\ref{fig-regressionsGAM} shows partial effect plots. They display
the relationship between the predictors ( `colgpa', `hsgpa', `study' ,
`work', `act') and the dependent variable. The plots show how each
predictor contributes to the dependent variable while allowing for
non-linear relationships.

The effect of college GPA is non-linear and strongly positive. As
college GPA increases, the smooth effect increases substantially. This
indicates that higher college GPAs are strongly associated with better
exam scores.

The effect of high school GPA appears linear and slightly positive. This
suggests that higher high school GPAs are associated with better exam
scores.

The effect of study is close to zero, similar to our OLS regression.

The effect of work is close to zero across all values, suggesting no
significant relationship between hours worked and exam scores in this
model.

The relationship between ACT score and exam scores is non-linear. The
effect increases initially and then plateaus around ACT scores of 25-30.
This indicates diminishing returns to very high ACT scores.

Comparing our OLS model and our GAM models; Both OLS and GAM models
suggest that hours studied per week does not have a significant impact
on exam scores. This is also true for hours worked per week. Both OLS
models and GAM models indicate a strong positive relationship between
College GPA, High school GPA and ACT score with Final exam scores.

\subsection{4. Machine learning methods}\label{machine-learning-methods}

This section is dedicated to using machine learning to further analyse
variable importance. Figure~\ref{fig-ml1} and Figure~\ref{fig-ml2} show
the variable importance from a random forest model. They use 2 metrics;
Mean decrease in gini, and Mean decrease in accuracy.

\begin{figure}

\centering{

\pandocbounded{\includegraphics[keepaspectratio]{ECON30520-Project_files/figure-pdf/fig-ml1-1.pdf}}

}

\caption{\label{fig-ml1}Variable importance (Accuracy)}

\end{figure}%

Figure~\ref{fig-ml1} ranks variables based on the mean decrease in
accuracy. This quantifies how much the prediction error (MSE) increases
if a variable is removed.

College GPA emerges as the most important predictor, with a significant
impact on model accuracy (over 50\%). ACT composite score, ACT math
score and high school GPA are also highly important. If the student had
taken calculus, ACT English score and math quiz score has moderate
importance. Hours studied, hours worked and attendance measures all show
minimal contributions toward model accuracy, meaning they have limited
predictive power. Age and if the mother or father had a college Degree
have almost no impact, meaning they are negligible predictors.

\begin{figure}

\centering{

\pandocbounded{\includegraphics[keepaspectratio]{ECON30520-Project_files/figure-pdf/fig-ml2-1.pdf}}

}

\caption{\label{fig-ml2}Variable importance (Gini)}

\end{figure}%

Figure~\ref{fig-ml2} ranks variables according to their mean decrease in
Gini Index. This is a measure of node impurity in classification trees.

College GPA is by far the most important predictor, making it crucial
for determining final exam scores. Act composite score and ACT math
score are the next most important, followed by high school GPA and
whether or not the student took calculus. Variables like gender, hours
worked, and attendance have lower importance, meaning their role is
minimal in predicting final exam scores. Age and attbad appear to have
negligible importance in reducing node impurity.

Our findings here align with the OLS and GAM models, where College GPA,
ACT composite score and High school GPA emerge as the most significant
predictors. Both the Gini and accuracy metrics point to similar variable
ranking, increasing confidence in the robustness of these results.

\subsection{5. Conclusion}\label{conclusion}

Across all models, College GPA is the most significant predictor of
final exam scores. Both OLS and GAMs have shown a strong positive
relationship. Diminishing returns were found at the highest GPA levels
through GAMs and Random forests. High school GPA was a moderately
important predictor. It contributed positively and linearly to scores.
Both of these factors underscore the importance of academic preparation
in final exam score for economics students. ACT composite score was also
consistently significant, with GAMs highlighting diminishing returns at
higher ACT levels. ACT math score showed strong predictive power in the
random forests model, showing the importance of mathematic aptitude when
under taking an economics course. Neither study hours, nor hours worked
showed significant relationships with exam scores. This prompts further
investigation to see if this is truly the case. Students who had taken
calculus performed better on average, as is shown by our OLS model and
its moderate importance in the Random Forest model. Gender, attendance,
parental education all exhibited weak effects. Gender displayed a small
positive effect in our OLS model, however it did not appear a
significant factor in either our GAM or random forest models.

\subsection*{References}\label{references}
\addcontentsline{toc}{subsection}{References}

\phantomsection\label{refs}
\begin{CSLReferences}{1}{0}
\bibitem[\citeproctext]{ref-turkey}
Erdem, Cumhur, Ismail Senturk, and Cem Kaan Arslan. 2008. {``The
Socioeconomic Determinants of the University Entrance Exam Scores in
Turkey.''} \emph{Int. J. Manage. Educ.} 2 (4): 357.

\end{CSLReferences}




\end{document}
